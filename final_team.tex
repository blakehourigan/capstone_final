\documentclass{article}

\usepackage{hyperref}
\usepackage{listings}
\usepackage{xcolor}
\usepackage{graphicx}
\usepackage{subcaption}
\usepackage{placeins}
\usepackage{pdfpages}

\graphicspath{{"pictures"}}

\lstset{
    basicstyle=\ttfamily\small,
    breaklines=true,
    numbers=left,
    numberstyle=\tiny,
    frame=single,
    keywordstyle=\color{blue},
    commentstyle=\color{gray},
    stringstyle=\color{orange},
}
\lstdefinelanguage{yaml}{
    morekeywords={true,false,null,y,n},      % Add keywords here
    sensitive=false,                        % Case insensitive
    morecomment=[l]{\#},                    % Line comments start with #
    morestring=[b]",                        % Double-quoted strings
    morestring=[b]',                        % Single-quoted strings
}

\setlength{\parskip}{1em}

\title{Chemical and Inventory Management System \\ SoFab Inks \\ Team 3}
\date{November 29, 2024}
\author{Hilton Benson, Blake Hourigan, CJ Johnstone, Maggie Jackey}

\begin{document}  
\maketitle
\clearpage
\tableofcontents
\clearpage

\section{Introduction\slash Executive Summary} 
\subsection{Introducing SoFab Inks}
SoFab Inks is a chemical manufacturing startup that was spun-out from the University of Louisville, Conn Center for Renewable Energy 
Research with support from the US DoE. SoFab inks focuses on accelerating the commercialization of Perovskite Solar Cells 
through the development and manufacturing of functionalized inks that improve cell efficiency, reduce module cost, and enable scalable 
manufacturing. \cite{sofabinks}
\subsection{SoFab Inks Inventory Management Problem}
SoFab Inks is doing important work in the field of solar cell technology, helping to drive humanity towards a cleaner, more energy 
abundant future. The problem, however, is that the team currently faces issues with managing inventory. These challenges prevent SoFab's
talented team from working on the most important aspects of their work. These challenges include expending valuable energy on menial 
tasks like locating inventory, managing a growing number of shipments manually, scouring inventory entries found across several 
Google Sheets or handwritten labels to pinpoint important product information, and more. 

To aid SoFab in these challenges, this semester Team 3 was tasked with developing a more efficient method of managing inventory items. 
To accomplish this, the team employed various existing software solutions, including database software, CRUD\footnote{Database term
for Create, Read, Update, Delete.} user interface software, and solutions to containerize this software together into one package. The 
team also developed custom software to provide features not available in the existing software solutions. 

CAC 3. Communicate effectively in a variety of professional contexts 

\section{System Description}
The following sections describe the specifications that were formulated as a result of communication between Team 3 and the SoFab Inks
team. 
\subsection{Needs Assessment\slash System Requirements}
Following assignment to this project, the team assembled and met virtually with the SoFab Inks team for brief self-introductions, 
an overview of the current issues facing SoFab, and to gain an initial insight of what the SoFab team was looking for in a
solution to these issues. 

The SoFab team described the current state of inventory management at the company which included problems such as shipments arriving 
to incorrect customers, an inability to pinpoint important information about products as they progressed through the manufacturing 
process, and difficult to track remaining volumes of chemical products. In these discussions, SoFab also emphasized the importance 
of a solution begin \textit{simple and easy-to-use}. As the team is primarily constructed of chemical engineers or business people, 
technology was not a strength. 

The company also made it clear that the ability to generate item labels for internal tracking of 
should be a priority. These labels would allow members of their team to easily scan a barcode to view the details of an item. Finally, 
it was also made clear that the company also desired the ability to generate shipment labels for their products as well. These labels 
would differ slightly from their internal tracking label counterparts with the inclusion of chemical hazard information. This information
would be included as a way to reduce harm for customers handling SoFab's chemicals upon receipt.

After this initial discussion, Team 3 began to brainstorm potential solutions to the problems presented. Immediately, 3 main 
pieces of software jumped out at the team as critical pieces to what would be the final product.

\begin{enumerate}
    \item \textbf{A Database System} - In order to move the company away from the usage of Google Sheets and toward a more efficient, 
        safe and redundant, user-friendly solution, Team 3 knew it would be necessary to select an existing database software solution.
        While the team was unsure of what specific solution would be chosen, it was sure that one of these solutions would be required.
    \item \textbf{A User-Friendly Database Interface} - While a database solution would be an incredible improvement on its own, it 
        would be useless to the SoFab team if there were not a simple and easy way to interact with the underlying data. Again, the team
        was unsure of what specific solution would be chosen, but a few requirements from discussions with SoFab were clear. 
        \begin{itemize}
            \item \textbf{Clean, Simple, Easy-to-Use} - As previously discussed, SoFab was clear that a simple and easy-to-use solution
                was of paramount importance for the day-to-day usage of the product. 
            \item \textbf{Free and Open-Source} - While this requirement was not mentioned in the initial discussions with SoFab, this 
                requirement jumped out as important to Team 3 because this would help to avoid incurring additional costs beyond the 
                development cost that SoFab had already paid. 
        \end{itemize}
    \item \textbf{Software to Generate Internal and Shipment Labels} - After discussing the need for barcode and label generation software,
        the team found that it would likely be necessary to build custom software to meet the needs of the client. The requirements were 
        quite specific, and would not be available in any existing commercial product. These requirements certainly would not be made 
        available in any \textit{free and open-source} software.
\end{enumerate}
\subsection{Initial System Specification}
\label{sec:init-sys-specs}
Following the identification of the 3 main categories of software that would comprise the solution to SoFab's inventory management 
problems, Team 3 dove into research of each individual component to identify optimal selections. 
\subsubsection{Selecting a Database}
Firstly, it was important for Team 3 to become familiar with existing solutions that were \textbf{free and open-source}. 
This required conducting research on computer-science related websites and forums, and more general forums like 
Reddit. While a site like Reddit may not always be the most reliable source of information, it could provide a general sense of what 
people feel about particular software and how easy it is work with. 

Many databases fulfilling the free and open-source requirement were identified, including: PostgreSQL, MySQL, MariaDB, and SQLite. As 
many options were available to select from, it was important to look beyond this requirement and investigate techinical specificaitions
of these solutions. During this time it was discovered that \textbf{PostgreSQL} had several technical benefits over more popular 
rivals. PostgreSQL boasts of being a fully \textbf{ACID} compliant database system. With features such as \textit{Write-Ahead Logging},
which writes transactions to a log file to avoid pushing a full table(s) every time a transaction is committed, \textit{Multi-version 
Concurrency Control} which the PostgreSQL documentation describes as 

\begin{quote}``This means that while querying a database each transaction 
sees a snapshot of data (a database version) as it was some time ago, regardless of the current state of the underlying data. 
This protects the transaction from viewing inconsistent data that could be caused by (other) concurrent transaction updates on the 
same data rows, providing transaction isolation for each database session.''\cite{postgresql-mvcc}
\end{quote} 

Furthermore, PostgreSQL utilizes its \textit{Write-Ahead Log} to implement \textit{Point-in-Time Recovery} without the need of 
complete backup. Additionally, since the Write-Ahead Log contains all transactions since the previous system backup, one can 
return to the exact state of \textbf{any} \textit{point in time} between the most recent backup version and the current version. 
Team 3 felt that these benefits would be incredibly beneficial for SoFab's new database, which would be the central hub containing 
the whereabouts and remaining inventory of their lab. This database would also contain important product information which if lost 
or damaged could have severe consequences for their business. 

Beyond technical considerations, ease of use was of great consideration when selecting a database, and luckily, PostgreSQL 
benefits from being very easy-to-use. PostgreSQL has an additional software called \textbf{PgAdmin4}, a project led by a core developer of 
PostgreSQL! This software integrates very well with PSQL\footnote{PSQL is short for PostgreSQL.}
This software provides a GUI that allows developers to interact with the database, run queries, add and edit tables, view ERDs
\footnote{ERD stands for entity relationship diagrams.} for tables, and much more. This would simplify team members development tasks
significantly and enable faster production. 

For the reasons stated above, PostgreSQL was selected as the database software of choice for this project. 
\subsubsection{Selecting an Interface}
While PostgreSQL would be a great choice to build a database for SoFab inks, this software alone - as stated previously - would be 
useless to the SoFab team by itself. Team 3 needed to produce a user interface that was simple, user-friendly, and would be capable 
of fulfilling all of SoFab's functional requirements. Team 3 scoured the internet, searching for a solution that would fulfil these 
requirements. Eventually, the open-source software Budibase was found. Upon investigation into Budibase, it was discovered that this 
software could be self-hosted, and if self-hosted, was \textit{free-to-use}. This immediately fulfilled two requirements, so the team
installed the software locally and began to interact with it to determine if it could fill the user-friendliness and functional 
requirements discussed previously. 

It was found that Budibase functions in a manner similar to that of website builders like ``Wix'' or ``Squarespace''. An architect
has the ability to connect to an existing database, 
then can select from existing templates based on the type of project at hand, and even use templates for types of pages to use. The 
architect can select between form-type templates where the user can edit or insert information about a row in a database table, or table-type 
templates where the user can view large amounts of data at a glance. 

Additionally, one can drag and drop pre-made components from the sidebar, and arrange them as they see fit. These components can 
connect to any data source in the database. Data sources include standard database tables and even custom queries to provide maximum 
customizability. 

While using a builder software like Budibase reduces the complexity in creating CRUD apps, they can pose a challenge in that all 
documentation and 
instructions for using the software provided strictly by Budibase themselves. The self-hosted Budibase community is not 
very large, so if an issue arises, an architect may be left to themselves to resolve it if that issue is not covered
in the existing documentation. 

Thankfully, however, the Budibase team has prioritized documentation, with instructions on how to use their many available 
components in various ways. They have created a page dedicated to the use of their platform, self-hosting instructions and guidance, 
component use, connecting the service to the database software in use, and much more. \cite{budibase-docs} The team also provides 
many instructional videos and how-tos on their website and on their company YouTube page. \cite{budibase-youtube}

\subsubsection{A Language for Custom Software}
After selections were made for database and user-interface software, discussions were held on the topic of how the final major piece 
of the software stack - the custom label printing interface - would be constructed. Team 3 understood SoFab's requirement for this 
deliverable - that the software create printable labels to place on their inventory - but Team 3 imposed additional requirements
that would support the team to fulfill the company's requirement. These requirements were identified as follows. 
\begin{enumerate}
    \item \textbf{Familiarity} - The most important requirement that was identified by Team 3 during these discussions was familiarity 
        among team members with various programming languages. Due to semester imposed time constraints, it would be important to hit 
        the ground running, without additional challenges of learning an unfamiliar programming language. A consensus was reached that 
        team members all had prior experience with \textit{Python}. This prior experience, paired with the languages intuitive syntax
        made Python an attractive choice with regard to this requirement. 
    \item \textbf{Community Support\slash Libraries} - Another important requirement concerning the selection of an optimal programming 
        language to develop this new software was community support and library availability. Once again, \textit{Python} was identified
        as a language with great strength in this aspect. Python is known for its extensive library support, with over 589,000 projects
        in the `pypi' Python package repositories. \cite{pypi} Packages were identified that would allow Team 3 to easily generate 
        barcodes, generate PDF outputs, and read from PostgreSQL databases. These features attributed to the appeal of Python for the 
        development of this deliverable. 
    \item \textbf{Reliability\slash Durability} - The two requirements previously discussed were found to be great strengths of the 
        Python language. However, an additional consideration for selection for this custom software was the durability of the language. 
        A reasonable argument could be made that the responsibility of ensuring reliability and durability of code lies with the 
        programmer, and not the language that a programmer may employ. However, humans have proven to be much less reliable than 
        machines in many domains with defined rules. 

        Recent developments in the programming domain such as the Rust language have built 
        this fact of reality into the fabric of the language. These such languages provide memory safety by default, meaning that the 
        programmer must have advanced knowledge of the language before performing potentially dangerous programmatic actions. Rust also 
        provides an excellent ecosystem of tools such as `Cargo' which both manages project dependencies and provides an easy method to
        run and test written code. Rust offers many benefits that Team 3 believed could be incredibly beneficial for a project that would 
        need to be consistent and reliable day-to-day at SoFab Inks.
    \item \textbf{Speed\slash Efficiency} - A final major consideration in choosing a programming language was the speed that the chosen
        language would provide after development. Performance is \textit{always} an important consideration, but this was  
        especially true with Team 3's plan to install many pieces of software on one machine on final deployment. On this point, Rust 
        again was found to have great advantages. Compared to Python, Rust is far more efficient, as it is 
        a compiled language. This means that at program run-time, no `interpretation' of the code is required. This essentially shifts much
        of the heavy computational lifting to the code `compilation' process, allowing the program to execute more efficiently at run-time. 
\end{enumerate}

Upon reviewing the major points of consideration, Team 3 decided that Familiarity and Comunity Support\slash Library availability were 
critical attributes in the lanuage selection process. While a language such as Rust provided significany boosts in execution time
and Reliability, both of which would be beneficial to the end user, it was decided that the team had too little familiarity with this 
language. Learning a new language as a team would increase the complexity of the project too greatly, and endanger Team 3's ability to
produce a satisfactory product on-time. It was for these reasons that Team 3 decided to select Python as the language of choice for the 
development of this additional software, pending approval by the SoFab team during the next meeting. 


(External design document) 
    EAC 1. Identify, formulate, and solve complex engineering problems by 
    applying principles of engineering, science, and mathematics and CAC 1. 
    Analyze a complex computing problem and to apply principles of computing 
    and other relevant disciplines to identify solutions) 

\subsection{Final Specifications}
Little changed with regard to the major required specifications of chosen software over the course of this project. While minor changes
were implemented in requirements with regard to specific data tracking points or tracking methods, these could all be accomodated within
the existing framework. For this reason the majority of the final technical specifications were set during the meeting that followed
Team 3's discussion of initial specifications for the system.

Team 3 met with the SoFab team to discuss the specifications that Team 3 had arrived at as results of the prior discussions. The team 
wanted to ensure that these specifications aligned with the needs and vision of the company. Team 3 proposed the previously discussed
database and user-interface solutions, with brief technical demonstrations showcasing the potential of these pieces of software as 
solutions to the inventory management problem. SoFab responded positively, with few notes with regard to the technical decisions reached 
by Team 3. 

For these reasons the following selections were made final as software solutions for this project.

\begin{itemize}
    \item \textbf{Database System} - PostgreSQL
    \item \textbf{User-Interface\slash CRUD Frontend} - Budibase
    \item \textbf{Programming Language for Custom Software} - Python
\end{itemize}
(finalized internal design document) 
    EAC 1. Identify, formulate, and solve complex engineering problems by 
    applying principles of engineering, science, and mathematics and CAC 1. 
    Analyze a complex computing problem and to apply principles of computing 
    and other relevant disciplines to identify solutions and CAC 6. Apply 
    computer science theory and software development fundamentals to produce 
    computing-based solutions) 

\subsection{System Diagrams} 
Detail all interfaces between the environment and the components 
    EAC 2. Apply engineering design to produce solutions that meet specified 
    needs with consideration of public health, safety, and welfare, as well as 
    global, cultural, social, environmental, and economic factors) 

\subsection{Hardware Overview Diagram} 
CAC 2. Design, implement, and evaluate a 
    computing-based solution to meet a given set of computing requirements in 
    the context of the program’s discipline) 

\subsection{Software Overview Diagram} 
CAC 2. Design, implement, and evaluate a 
    computing-based solution to meet a given set of computing requirements in the 
    context of the program’s discipline) 

\subsection{Economical, Technical, and Time Constraints}
Over the course of this project, several constraints were imposed and revealed to Team 3 that limited the available tools for this 
project. These constraints fell under three main categories, each of which will be discussed in its respective section below. 
\subsubsection{Economic Constraints}
Firstly, Team 3 experienced economic constraints, financially limiting software solutions available as tools for this project. Team 3 
was limited economically primarily due to the identified requirement that utilized software be \textit{free and open-source}. As 
discussed previously, this requirement was identified to avoid incurring subscription costs charged to SoFab following development of 
the project solution. This limitation significantly limited the software tools available to Team 3 for development. 

Additionally, Team 3 was constrained economically due to the inability to provide development hardware to the SoFab team upon project
completion. Unbeknownst to Team 3 early in the semester, hardware that was purchased to facilitate development of the project would be 
unable to be transferred to the on-site location upon completion. Luckily, the university approved the request to transfer necessary 
hard drives containing critical database information upon project completion, but did not approve this request for other hardware 
necessary for this projects successful completion, such as a barcode scanner or label printer. For this reason the team was economically 
constrained and encouraged to develop a solution to SoFab's inventory management problem on the smallest budget the team could manage. 

Similarly to the previous constraint, meeting this constraint would be important to ensure that SoFab would incur the smallest possible
cost upon project completion. Team 3 wanted to deliver a project that contained as much of the desired functionality as possible, while
avoiding extra purchases like barcode scanners or product label printers. 

The team will expand on the specific solutions implemented to meet project requirements in the face of these constraints, however, the 
main ways in which economic constraints were addressed follow below. 

\begin{enumerate}
    \item Free and Open-Source Software
    \item Software Implementation of Hardware Features\slash Creative Use of Existing Hardware
\end{enumerate}

\subsubsection{Technical Constraints}
The next primary factor which constrained Team 3 during the development of the solution their technical ability and technological 
constraints. The limitations in technical ability discussed here coincide with time constrains, however technical ability specifically 
will be discussed in this section. As discussed in Section \ref{sec:init-sys-specs}, Team 3 was limited technically in that all team 
members had the most familiarity with one programming language - \textit{Python}. This imposed several technical limitations including 
reliability and efficiency that were traded for familiarity, ease-of-use, and library availability. The Rust programming language's 
memory safety features, combined with its status as a compiled language held much in the way of potential benefits for the development 
of this project. However, due to Team 3's unfamiliarity with the language, these potential benefits became limitations by use of Python, 
which faces challenges both with reliability and efficiency. 

Additionally, Team 3 was consrained technologically by installation hardware. Upon completion, this project's software stack would 
be installed onto SoFab's main laboratory computer. This software stack would act as a server on the local area network to field requests
from mobile devices for ease-of-use and convenience. It was for this reason that Team 3 was obligated to be mindful of resource usage 
regarding installed software, and aware of the computational cost of adding further software. The installation site would be used for 
other tasks besides this inventory management system, and therefore the system needed to function efficiently to avoid slowing down 
other operations and tasks conducted on the lab computer. 

Specific action taken to avoid concerns regarding these constraints is discussed in Section \ref{sec:det-imp}, however a general outline
is provided here. 
\begin{enumerate}
    \item Mindfulness Regarding Existing Lab Hardware
    \item Implementing Only Critical Software Solutions
    \item Writing Efficient Python Code
\end{enumerate}
\subsubsection{Time Constraints}
The main factor which imposed constraints on Team 3 during the Fall 2024 semester was time. The team had approximately 3 months to develop
a solution that would satisfy the requirement of the SoFab team, and this meant that Team 3 needed to act fast to develop their product.
This deadline itself imposed constraints on the team, such as how often and with what intensity the team needed to act, however it was 
also the primary factor in imposing technical limitation. Team 3 simply had no time to learn a new programming language 
comprehensively 
prior to implementing a solution. This was the driving factor in the selection of a familiar and well-supported programming language 
such as \textit{Python.} 

Furthermore, meeting dates with the SoFab team imposed constraints in that signicant progress was expected of the team approximately
every 2 weeks, when meetings would occur. After one of these meetings had concluded, the team had 2 weeks to implement early versions
of the discussed features to be showcased at the next meeting. This imposed an intensity that Team 3 needed to work with to be prepared
for every meeting. 

\section{Detailed Implementation} 
\label{sec:det-imp}
The below sections include the specific actions and steps taken by Team 3 to develop the solution to SoFab Inks inventory management
problem with project requirements and constraints accounted for. 
\subsection{Hardware Detailed Implementation}
While the solution developed over the course of this project was constructed primarily of software components, some hardware components 
were required to meet and exceed the requirements of SoFab Inks. The primary hardware components are covered in detail below.
\subsubsection{A Home for A Database}
As discussed in previous sections, in order to develop software that improved significantly on SoFab team members experience of managing
inventory, a database system was required. Once this was decided, the main point of emphasis was how to implement this requirement as 
safely as possible. Team 3 wanted to ensure with certainty that SoFabs sensitive and irreplaceable company data would be both secure 
and redundant to a avoid a catastrophic data loss scenario. This requirement however was not without mild cost. 

The security of the database was a matter for the software implementation, however, redundancy required multiple storage locations to 
to ensure multiple independent copies of the database are available at any given time. For this reason Team 3 decided to purchase 2 
1 Terabyte solid-state storage drives, intended to each contain separate copies of the database. One of these disks would contain the 
current working version of the database at all times, while the other drive would contain backups that would be conducted on a daily 
basis. The team could even write a script in such a way to keep record of each backup instead of overwriting the backup each day with the 
new copy. This would allow the team to inspect the database at any given point in its history if required. 

Furthermore, investing in these drives at the beginning of the project life-cycle improved the experience of the transition to the 
installation site significantly. Because the team was able to develop the project on the purchased SSDs,\footnote{SSD is a computational
term for Solid State storage Disk.} when the project was transferred on-site, the team could simply bring the drive containing the up-to-date
software stack, point the lab computer to this disk, and have the project up-and-running. 
\subsubsection{A Barcode Scanning Interface} 
An additional piece of hardware needed to meet the full requirements of the SoFab team was a barcode\slash QR code label scanner that 
would allow the team to simply scan product to view their corresponding inventory details. While the SoFab team would be unable to keep 
the scanner that Team 3 purchased for development due to university rules, this product was inexpensive and in SoFab Inks' reasonble price
range. The team conducted research into barcode scanning technology to discover what attributes made a good scanner. 

Team 3 found that most barcode capable hardware scanners could double as QR code scanners, with little differences in feature sets between
available models. For this reason the team sought an affordable wireless model that would allow SoFab to walk around their laboratory and 
scan products instead of being required to retrieve products first. The implementation of this barcode scanning technology required little
beyond purchasing the product. Once received, Team 3 could simply plug the scanner's bundled USB dongle into any computer, scan a barcode 
or QR code while inside a text field, and the scanner would take care of the rest. This purchase made the development of this 
feature simple and instantly partially fulfilled a requirement set out by SoFab Inks for their inventory management system. 
\subsection{Software Detailed Implementation}
The vast majority of effort expended by members of Team 3 this semester was in the domain of software. The team worked to utilize a 
combination of existing and custom-built software to provide an inventory management system that would satisfy requirements laid out 
by SoFab Inks. The individual pieces of software that played major roles in constructing this solution are discussed at length in the 
below sections.
\subsubsection{A Home Development Server}
Due to the nature of the software being developed, Team 3 identified a need to create a centralized server that would allow members to 
access the software from any location at any time. During the early weeks of development, the team was using existing tools and 
services to develop a solution such as PostgreSQL and Budibase instead of writing custom code. Development with these services would 
benefit substantially from running on a centralized server that could be modified by any team member. This would be far more efficient 
than developing solutions individually and compiling them into one solution at a later time. 

An added benefit of creating a centralized server where all development would take place was that this server would closely mimic the 
conditions of final installation. Upon installation, the software stack would be deployed on a single computer that would serve all 
clients on a local area network. This experience would allow the members of Team 3 to work with an experience closer to that of an end 
user. This allowed team members to experience bugs that would appear at the installation site and fix them as they arose. 

To construct a home server that would host all services required for this project and allow all Team 3 members access, a computer 
was set-up at a members home and given a static IP. This static IP would allow DNS services and the home router to find a constant path 
to the services to be accessed. Next, it was time to install all the existing services that would be required to develop the project. This 
software will be discussed at length in Section \ref{sec:docker}. All services were installed and initialized. 

Next, a Nginx server 
running on another computer was configured as a reverse-proxy. A reverse-proxy is a service that acts as an intermediary for servers or 
services running on a local network. A reverse proxy intakes network requests and forwards them to the server where they are hosted. 
The benefits that a reverse-proxy offers to users is that only one 
port is required to be exposed to the internet. This reduces the number of open ports on a network, offering security benefits. Additionally,
reverse-proxies can be configured to use HTTPS,\footnote{Hypertext Transfer Protocol \textit{Secure.}} also enhancing the security of data 
\textit{transmission}. Finally, reverse-proxies can also provide load balancing services, reducing strain on any one server on the 
network. 

\subsubsection{Docker-Compose}
\label{sec:docker}
Docker - and its supplemental software Docker-Compose and Docker Desktop - played a pivotal role during the development of this project.
Docker is a software that allows applications to be packed into `containers'. This allows these pieces of software to run in isolation from 
other software, be more easily configured, and be isolated from the host system network which can enhance software security. While Docker 
containers share similarities with classical virtual machines in that they provide environments that are isolated from the host machine, 
Docker containers differ in a few major respects that are important to note in this context. 

Virtual machines are known for the performance penalty that accompanies their use. This is because virtual machines are tasked with 
emulating an entire operating system, which is a very complicated and resource intensive task. Docker containers work to alleviate 
this performance problem by \textit{sharing the host's operating system kernel}. This allows Docker to focus on the application 
software itself, without the need to emulate existing operating system features. 

Expanding on these benefits, Docker-Compose provides the ability to roll multiple Docker containers into one centralized script, 
where all software to be used for a given project can be managed at once. One can specify version numbers, customize network configurations,
assign storage mounts,\footnote{Storage mounts are locations in a computer where files are located.}, allocate extra resources,
\footnote{An example of an extra resource would be allocating GPU resources for a container.} and more. 

The incredible benefits that 
this provides when developing an application to be deployed on another system cannot be overstated. What Docker-Compose allowed Team 3 
to do over the course of this project is to develop a software stack with versions of software that were compatible with one another. The
team installed this software on a Windows 11 machine to mimic the operating system in use at the site of eventual installation, reducing 
possible complications with system migration. Using these methods, Team 3 could be confident when the time for system 
migration came, that all software would be compatible and work together smoothly, keeping all existing data intact. 

This technology removed a great burden from Team 3, and the effort placed into creating a stable software stack through Docker-Compose 
was rewarded upon migration to SoFab's laboratory in late November 2024. This will be explained in greater detail in a later section.
The software that was integrated into this Docker-Compose stack included the database software PostgreSQL, the database management 
software PgAdmin4, and database frontend design software Budibase. 

Most developers that provide the ability to deploy an application via Docker provide example Docker-Compose scripts defining the 
critical application configurations and environment variables to deploy the application. This was very beneficial to Team 3 for the 
development of this software stack, as examples for deploying PostgreSQL, PgAdmin4, and Budibase were found, accelerating deployment.
\cite{dockerhub-postgres} \cite{dockerhub-pgadmin} \cite{budibase-docker-compose}A sample of the final Docker-Compose Script is 
included below. 

\begin{lstlisting}[language=yaml]
 postgres:
   image: postgres:16
   volumes:
     - type: bind
       source: "D:/Docker Data/postgres_data"
       target: /var/lib/postgresql/data
   ports:
     - "5432:5432"
   restart: always
   container_name: postgres
   # set shared memory limit when using docker-compose
   shm_size: 128mb
   # or set shared memory limit when deploy via swarm stack
   environment:
     POSTGRES_PASSWORD: {POSTGRES_PASSWORD}
   extra_hosts:
     - "--add-host host.docker.internal:host-gateway"

 pgadmin4:
       image: elestio/pgadmin
       restart: always
       environment:
         PGADMIN_DEFAULT_EMAIL: ${ADMIN_EMAIL}
         PGADMIN_DEFAULT_PASSWORD: ${ADMIN_PASSWORD}
         PGADMIN_LISTEN_PORT: 8080
       ports:
         - "0.0.0.0:8080:8080"
       volumes:
         - pg_admin_data:/var/lib/pgadmin

volumes: 
  pg_admin_data:
\end{lstlisting}
\subsubsection{PostgreSQL}
Once the Docker-Compose script was defined and all software was deployed on the home server machine and exposed to all Team 3 members, 
application specific configuration began for PostgreSQL. User accounts were created (Figure \ref{fig:psql_users}) for each team member,
and credentials were distributed. Team 3 then initialized the database for the project (Figure \ref{fig:psql_dbs}), and set 
user privileges. PosgreSQL's user creation tool was utilized to add an additional account `database dev' which would be given privileges
to read, add, create, and delete on the database. These permissions could then be applied recursively to each user account, streamlining
the permissions process. 

\begin{figure}[h!]
    \centering
    \begin{subfigure}[b]{\textwidth}
        \centering
        \includegraphics[width=10cm]{"postgres_users.png"}
        \caption{Viewing existing users in the PSQL CLI interface.}
        \label{fig:psql_users}
    \end{subfigure}
    \begin{subfigure}[b]{\textwidth}
        \centering
        \includegraphics[width=10cm]{"postgres_dbs.png"}
        \caption{Viewing existing tables in the PSQL CLI interface.}
        \label{fig:psql_dbs}
    \end{subfigure}
    \caption{}
    \label{fig:posgres_cli}
\end{figure}
\FloatBarrier

\subsubsection{PgAdmin4}
With the database initialized, the team could begin work on constructing a relational database in PSQL\footnote{PSQL stands for PostgreSQL.}
that modeled the daily workflow in the SoFab Inks lab. However, Team 3 felt that it would be quite inefficient to develop in the 
PSQL CLI\footnote{CLI stands for Command Line Interface in a computer terminal}. Members felt that it would be difficult to grasp the 
relations in a database and quickly view the attributes of a database with this interface, especially due to the length of time members
had gone without working on database systems. Unfamilarity would lead to frustration and confusion through this method of interaction. 
For this reason members of the team began to seek out away to interact with a database system with a graphical user interface. 

Quickly, members found that there existed a GUI\footnote{Graphical User Interface.} tool build specifically for PSQL called `PgAdmin4'. 
Best of all, this software was developed by a core member of the PSQL team! PgAdmin4 boasted features such as graphical management of 
existing attributes including databse schemas, tables, types, trigger functions, and more. This would allow the team to manage the 
database visually in a way that would allow members to quickly and easily learn the tools available at their disposal in PSQL. Additionally,
this tool allowed for the execution of database queries, giving developers all the power of CLI PSQL in addition to the GUI tools. 

PgAdmin4 was deployed in the projects Docker-Compose script and deployed alongside PSQL. Once deployed, user accounts were created and 
configured to access the PSQL database as these accounts. At this stage Team 3 could begin constructing the database while avoiding
the overwhelming nature of a CLI tool to interface with the database system. Examples of the deployed instance after the insertion of 
tables can be found in Figures \ref{fig:pg_admin_tables}, and \ref{fig:pg_admin_queries}. More images can be found on PgAdmin4's webpage.
\cite{pgadmin-screenshots}

\begin{figure}[h!]
    \centering
    \begin{subfigure}[b]{.45\textwidth}
        \centering
        \includegraphics[width=6cm]{"pg_admin.png"}
        \caption{Viewing existing tables in PgAdmin interface.}
        \label{fig:pg_admin_tables}
    \end{subfigure}
    \begin{subfigure}[b]{.45\textwidth}
        \centering
        \includegraphics[width=6cm]{"pg_admin2.png"}
        \caption{Running PostgreSQL queries inside PgAdmin4.}
        \label{fig:pg_admin_queries}
    \end{subfigure}
    \caption{}
    \label{fig:pg_admin_figs}
\end{figure}
\FloatBarrier

In regard to the construction of the database, the same general process was followed for three primary forms of inventory in use 
at the SoFab team's lab. These consisted in `Product Inventory' which contained information about the company's products, 
`General Inventory' which contained general lab items such as beakers and shipping boxes, and `Chemical Inventory' which the 
company used as ingredients to their products. SoFab desired to have information about these items at each stage in their life-cycle. 
While this cycle changed depending on the type of inventory an item originated from, the process of creating tables to accomplish this
effect would remain the same. 

The general process follows below.
\begin{enumerate}
    \item \textbf{Create a Table for the \textit{Item}} - A table would be created for the type of inventory item, containing any 
        \textit{static}, or, constantly present, information about this type of item. This item would also contain the \textit{ID}, 
        which would be used to link each row, which represented one item, to related rows in other tables. 
    \item \textbf{Create a Table for Each Life \textit{Stage}} - Next, team members would construct tables that represented a stage in a 
        product's life. For example, the `Product' inventory may have product table containing the ID of this product, chemicals used to 
        create this product, and in what amount, and other related information. Then, for each stage in a manufacturing process, 
        tables could be generated that would track specific information about actions performed on the item at this step. This was an
        important attribute, as this would give SoFab the ability to easily review the histories of successful batches, and be able to 
        recreate the product as closely as possible. 

        Similar steps would be followed for other categories of items, tracking information at every step related to an item that was 
        necessary as instructed by SoFab team members. 
    \item \textbf{Building Table \textit{Relationships}} - An important last step in the creation of the database is creating links 
        that connect tables together in terms of their relation to one another. This was generally accomplished by linked item IDs to 
        each other via `foreign keys', that helped the database understand the connections between different tables. A major benefit of 
        this relation creation is that it enables automatic generation of ERD\footnote{Entity Relationship Diagram.} diagrams. These 
        diagrams aid developers by visually representing database tables. An example of an ERD for this database can be found in 
        Figure \ref{fig:pg_admin_erd}.
\end{enumerate}

\begin{figure}[h!]
    \centering
    \includegraphics[width=12cm]{"pg_admin_erd.png"}
    \caption{ERDs help developers understand how tables are related.}
    \label{fig:pg_admin_erd}
\end{figure}
\FloatBarrier

\subsubsection{Budibase}
\begin{figure}[h!]
    \centering
    \includegraphics[width=12cm]{"budibase.png"}
    \caption{Budibase home page. Global search that searches all types of inventory via QR code or direct ID entry.}
    \label{fig:budibase_home}
\end{figure}
\begin{figure}[h!]
    \centering
    \includegraphics[width=12cm]{"budibase2.png"}
    \caption{Budibase general inventory page. Keep track of general lab items.}
    \label{fig:budibase_gen_inv}
\end{figure}
\begin{figure}[h!]
    \centering
    \includegraphics[width=12cm]{"budibase3.png"}
    \caption{Keep track of customer information with Budibase. View customer information or view all orders originating from a
    customer}
    \label{fig:budibase_customers}
\end{figure}

\FloatBarrier
\subsubsection{Python Database Insertion\slash Label Generation}
\begin{lstlisting}[language=Python]
class App(Tk):
    def __init__(self, controller):
        # initializing the Tk class instance
        super().__init__()

        self.title("SoFab Inventory Managment System")

        self.style = Style(theme="darkly")

        self.controller = controller
        self.controller.set_view(self)

        self.notebook = ttk.Notebook()
        self.notebook.pack(fill="both", expand=True)
        self.notebook.bind("<<NotebookTabChanged>>", self.on_tab_selection)

        # filling the notebook (top tabs) with frames

        item_type_tables = self.controller.get_item_type_tables()
\end{lstlisting}
\begin{lstlisting}[language=Python]
class HazardPrecautionFrame(tk.Frame):
    def __init__(self, parent, controller, warning_dict, images=False):
        super().__init__(parent)
\end{lstlisting}
\begin{lstlisting}[language=Python]
def generate_checkboxes(self, images=False):
    self.checkboxes_frame = tk.Frame(self)
    self.checkboxes_frame.grid(row=0, column=0)

    for item in self.warning_items:
        if images:
            image = Image.open(item[1])
            resized_image = image.resize((100, 100), Image.LANCZOS)
            image = ImageTk.PhotoImage(resized_image)
            item = item[0]
        else:
            image = None
        var = tk.BooleanVar()
        checkbox = ttk.Checkbutton(
            self.checkboxes_frame,
            text=item,
            image=image,
            compound="left",
            variable=var,
            command=lambda var=var: self.parent.update_text_box(),
        )
        checkbox.image = image

        checkbox.pack(anchor="w", fill="x")
\end{lstlisting}

\subsubsection{Deployment to SoFab Labs}

EAC 1. Identify, formulate, and solve complex engineering problems by 
applying principles of engineering, science, and mathematics and EAC 2. 
Apply engineering design to produce solutions that meet specified needs with 
consideration of public health, safety, and welfare, as well as global, cultural, 
social, environmental, and economic factors, and CAC 6. Apply computer 
science theory and software development fundamentals to produce 
computing-based solutions) 

\section{Test/Evaluation Experimental Procedure and Analysis of Results} 
EAC 6. Develop and conduct appropriate experimentation, analyze and interpret 
data, and use engineering judgment to draw conclusions) 

\section{Societal Impact of Project\slash Legal and Ethical Considerations} 
include legal and ethical considerations 

CAC 4. Recognize professional responsibilities and make informed judgments in 
computing practice based on legal and ethical principles) 

\section{Contribution of Project to Society\slash Expected Effects}
CAC 4. Recognize professional responsibilities and make informed judgments in 
computing practice based on legal and ethical principles) 

\section{Engineering Standards, Constraints, and Security}
EAC 1. Apply engineering design to produce solutions that meet specified needs with consideration of public health, 
safety, and welfare, as well as global, cultural, social, environmental, and economic 
factors) 
\section{Conclusions}

\section{Recommendations for Future Work}

\begin{thebibliography}{99}
\bibitem{sofabinks} SofaBinks. (n.d.). 
    About SofaBinks. Retrieved November 29, 2024, from \url{https://www.sofabinks.com/about}
\bibitem{postgresql-mvcc} PostgreSQL Global Development Group, \textit{MVCC: Multi-Version Concurrency Control}, 
    PostgreSQL 7.1 Documentation, \url{https://www.postgresql.org/docs/7.1/mvcc.html}, Accessed on: November 27, 2024.
\bibitem{budibase-docs} 
Budibase Team, 
\textit{What is Budibase?}, Budibase Documentation, 
\url{https://docs.budibase.com/docs/what-is-budibase}, 
accessed on: November 28, 2024.

\bibitem{budibase-youtube} 
Budibase, 
\textit{Budibase YouTube Channel}, YouTube, 
\url{https://www.youtube.com/@Budibase}, 
accessed on: November 28, 2024.
\bibitem{pypi} 
Python Software Foundation, 
\textit{The Python Package Index (PyPI)}, 
\url{https://pypi.org/}, 
accessed on: November 29, 2024.
\bibitem{budibase-docker-compose} Budibase, \textit{Docker Compose File for Hosting}, 
GitHub Repository, \url{https://raw.githubusercontent.com/Budibase/budibase/master/hosting/docker-compose.yaml}, 
Accessed on: November 27, 2024.
\bibitem{dockerhub-postgres} PostgreSQL Maintainers, 
\textit{PostgreSQL Docker Official Image}, Docker Hub, 
\url{https://hub.docker.com/_/postgres}, 
Accessed on: November 27, 2024.
\bibitem{dockerhub-pgadmin} Elestio, 
\textit{pgAdmin Docker Image}, Docker Hub, 
\url{https://hub.docker.com/r/elestio/pgadmin}, 
Accessed on: November 27, 2024.
\bibitem{pgadmin-screenshots} pgAdmin Development Team, 
\textit{pgAdmin Screenshots}, pgAdmin Official Website, 
\url{https://www.pgadmin.org/screenshots/#4}, 
Accessed on: December 1, 2024.
\end{thebibliography}


\addcontentsline{toc}{section}{References}

\newpage
\section*{Appendices} % Section without number
\addcontentsline{toc}{section}{Appendices} % Add it manually to the TOC
\appendix
\section{Customer Contact Information}
\section{Data Sheets}
\section{Additional Drawings and Diagrams}
\section{Source Code} 
\section{Experimental and/or Simulation Test Results} 
\section{Software Installation Instructions} 
\section{User Manual} 
\section{Quotes, Including Ordering Information} 
\section{White Papers} 


\end{document}
