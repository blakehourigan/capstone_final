\documentclass{article}

\usepackage{hyperref}
\usepackage{listings}
\usepackage{xcolor}
\usepackage{graphicx}
\usepackage{subcaption}
\usepackage{placeins}

\graphicspath{{"pictures"}}

\lstset{
    basicstyle=\ttfamily\small,
    breaklines=true,
    numbers=left,
    numberstyle=\tiny,
    frame=single,
    keywordstyle=\color{blue},
    commentstyle=\color{gray},
    stringstyle=\color{orange},
}
\lstdefinelanguage{yaml}{
    morekeywords={true,false,null,y,n},      % Add keywords here
    sensitive=false,                        % Case insensitive
    morecomment=[l]{\#},                    % Line comments start with #
    morestring=[b]",                        % Double-quoted strings
    morestring=[b]',                        % Single-quoted strings
}

\setlength{\parskip}{1em}

\title{Chemical and Inventory Management System \\ SoFab Inks \\ Team 3}
\date{November 29, 2024}
\author{Hilton Benson, Blake Hourigan, CJ Johnstone, Maggie Jackey}

\begin{document}  
\maketitle
\clearpage
\tableofcontents
\clearpage

\section{Introduction\slash Executive Summary} 
\subsection{Introducing SoFab Inks}
SoFab Inks is a chemical manufacturing startup that was spun-out from the University of Louisville, Conn Center for Renewable Energy 
Research with support from the US DoE. SoFab inks focuses on accelerating the commercialization of Perovskite Solar Cells 
through the development and manufacturing of functionalized inks that improve cell efficiency, reduce module cost, and enable scalable 
manufacturing. \cite{sofabinks}
\subsection{SoFab Inks Inventory Management Problem}
SoFab Inks is doing important work in the field of solar cell technology, helping to drive humanity towards a cleaner, more energy 
abundant future. The problem, however, is that the team currently faces issues with managing inventory. These challenges prevent SoFab's
talented team from working on the most important aspects of their work. These challenges include expending valuable energy on menial 
tasks like locating inventory, managing a growing number of shipments manually, scouring inventory entries found across several 
Google Sheets or handwritten labels to pinpoint important product information, and more. 

To aid SoFab in these challenges, this semester Team 3 was tasked with developing a more efficient method of managing inventory items. 
To accomplish this, the team employed various existing software solutions, including database software, CRUD\footnote{Database term
for Create, Read, Update, Delete.} user interface software, and solutions to containerize this software together into one package. The 
team also developed custom software to provide features not available in the existing software solutions. 

CAC 3. Communicate effectively in a variety of professional contexts 

\section{System Description}
The following sections describe the specifications that were formulated as a result of communication between Team 3 and the SoFab Inks
team. 
\subsection{Needs Assessment\slash System Requirements}
Following assignment to this project, the team assembled and met virtually with the SoFab Inks team for brief self-introductions, 
an overview of the current issues facing SoFab, and to gain an initial insight of what the SoFab team was looking for in a
solution to these issues. 

The SoFab team described the current state of inventory management at the company which included problems such as shipments arriving 
to incorrect customers, an inability to pinpoint important information about products as they progressed through the manufacturing 
process, and difficult to track remaining volumes of chemical products.


\begin{enumerate}
    \item A Database System
    \item A User-Friendly Database Interface
    \item Software to Generate Internal and Shipment Labels
\end{enumerate}
\subsection{Initial System Specification}
(External design document) 
    EAC 1. Identify, formulate, and solve complex engineering problems by 
    applying principles of engineering, science, and mathematics and CAC 1. 
    Analyze a complex computing problem and to apply principles of computing 
    and other relevant disciplines to identify solutions) 

\subsection{Final Specifications}
(finalized internal design document) 
    EAC 1. Identify, formulate, and solve complex engineering problems by 
    applying principles of engineering, science, and mathematics and CAC 1. 
    Analyze a complex computing problem and to apply principles of computing 
    and other relevant disciplines to identify solutions and CAC 6. Apply 
    computer science theory and software development fundamentals to produce 
    computing-based solutions) 

\subsection{System Diagrams} 
Detail all interfaces between the environment and the components 
    EAC 2. Apply engineering design to produce solutions that meet specified 
    needs with consideration of public health, safety, and welfare, as well as 
    global, cultural, social, environmental, and economic factors) 

\subsection{Hardware Overview Diagram} 
CAC 2. Design, implement, and evaluate a 
    computing-based solution to meet a given set of computing requirements in 
    the context of the program’s discipline) 

\subsection{Software Overview Diagram} 
CAC 2. Design, implement, and evaluate a 
    computing-based solution to meet a given set of computing requirements in the 
    context of the program’s discipline) 

\subsection{Economical, Technical, and Time Constraints}  

\section{Detailed Implementation} 

\subsection{Hardware Detailed Implementation}

\subsection{Software Detailed Implementation}
\subsubsection{Docker-Compose}
\subsubsection{PostgreSQL}
\subsubsection{PgAdmin4}
\begin{figure}[h!]
    \centering
    \begin{subfigure}[b]{.45\textwidth}
        \centering
        \includegraphics[width=5cm]{"pg_admin.png"}
        \caption{Viewing existing tables in PgAdmin interface.}
        \label{fig:pg_admin_tables}
    \end{subfigure}
    \begin{subfigure}[b]{.45\textwidth}
        \centering
        \includegraphics[width=5cm]{"pg_admin2.png"}
        \caption{Running PostgreSQL queries inside of PgAdmin4.}
        \label{fig:pg_admin_queries}
    \end{subfigure}
    \caption{}
    \label{fig:pg_admin_figs}
\end{figure}
\subsubsection{Budibase}
\begin{figure}[h!]
    \centering
    \includegraphics[width=12cm]{"budibase.png"}
    \caption{Budibase home page. Global search that searches all types of inventory via QR code or direct ID entry.}
    \label{fig:budibase_home}
\end{figure}
\begin{figure}[h!]
    \centering
    \includegraphics[width=12cm]{"budibase2.png"}
    \caption{Budibase general inventory page. Keep track of general lab items.}
    \label{fig:budibase_gen_inv}
\end{figure}
\begin{figure}[h!]
    \centering
    \includegraphics[width=12cm]{"budibase3.png"}
    \caption{Keep track of customer information with Budibase. View customer information or view all orders originating from a
    customer}
    \label{fig:budibase_customers}
\end{figure}

\FloatBarrier
\subsubsection{Python Database Insertion\slash Label Generation}
\begin{lstlisting}[language=Python]
class App(Tk):
    def __init__(self, controller):
        # initializing the Tk class instance
        super().__init__()

        self.title("SoFab Inventory Managment System")

        self.style = Style(theme="darkly")

        self.controller = controller
        self.controller.set_view(self)

        self.notebook = ttk.Notebook()
        self.notebook.pack(fill="both", expand=True)
        self.notebook.bind("<<NotebookTabChanged>>", self.on_tab_selection)

        # filling the notebook (top tabs) with frames

        item_type_tables = self.controller.get_item_type_tables()
\end{lstlisting}
\begin{lstlisting}[language=Python]
class HazardPrecautionFrame(tk.Frame):
    def __init__(self, parent, controller, warning_dict, images=False):
        super().__init__(parent)
\end{lstlisting}
\begin{lstlisting}[language=Python]
def generate_checkboxes(self, images=False):
    self.checkboxes_frame = tk.Frame(self)
    self.checkboxes_frame.grid(row=0, column=0)

    for item in self.warning_items:
        if images:
            image = Image.open(item[1])
            resized_image = image.resize((100, 100), Image.LANCZOS)
            image = ImageTk.PhotoImage(resized_image)
            item = item[0]
        else:
            image = None
        var = tk.BooleanVar()
        checkbox = ttk.Checkbutton(
            self.checkboxes_frame,
            text=item,
            image=image,
            compound="left",
            variable=var,
            command=lambda var=var: self.parent.update_text_box(),
        )
        checkbox.image = image

        checkbox.pack(anchor="w", fill="x")
\end{lstlisting}

EAC 1. Identify, formulate, and solve complex engineering problems by 
applying principles of engineering, science, and mathematics and EAC 2. 
Apply engineering design to produce solutions that meet specified needs with 
consideration of public health, safety, and welfare, as well as global, cultural, 
social, environmental, and economic factors, and CAC 6. Apply computer 
science theory and software development fundamentals to produce 
computing-based solutions) 

\section{Test/Evaluation Experimental Procedure and Analysis of Results} 
EAC 6. Develop and conduct appropriate experimentation, analyze and interpret 
data, and use engineering judgment to draw conclusions) 

\section{Societal Impact of Project\slash Legal and Ethical Considerations} 
include legal and ethical considerations 

CAC 4. Recognize professional responsibilities and make informed judgments in 
computing practice based on legal and ethical principles) 

\section{Contribution of Project to Society\slash Expected Effects}
CAC 4. Recognize professional responsibilities and make informed judgments in 
computing practice based on legal and ethical principles) 

\section{Engineering Standards, Constraints, and Security}
EAC 1. Apply engineering design to produce solutions that meet specified needs with consideration of public health, 
safety, and welfare, as well as global, cultural, social, environmental, and economic 
factors) 
\section{Conclusions}

\section{Recommendations for Future Work}

\begin{thebibliography}{99}
\bibitem{sofabinks} SofaBinks. (n.d.). 
    About SofaBinks. Retrieved November 29, 2024, from \url{https://www.sofabinks.com/about}
\end{thebibliography}

\addcontentsline{toc}{section}{References}

\newpage
\section*{Appendices} % Section without number
\addcontentsline{toc}{section}{Appendices} % Add it manually to the TOC
\appendix
\section{Customer Contact Information}
\section{Data Sheets}
\section{Additional Drawings and Diagrams}
\section{Source Code} 
\section{Experimental and/or Simulation Test Results} 
\section{Software Installation Instructions} 
\section{User Manual} 
\section{Quotes, Including Ordering Information} 
\section{White Papers} 


\end{document}
